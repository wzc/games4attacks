\documentclass[a4paper,11pt]{article}
\usepackage[left=.7in, right=.7in, top=.7in, bottom=.7in]{geometry}
\usepackage{xcolor,minted,fourier,palatino}
\usemintedstyle{vs}
\setminted[haskell]{fontfamily=tt}
\usepackage[T1]{fontenc}
\renewcommand{\baselinestretch}{1.12}
\newcommand{\comment}[1]{\textcolor{red}{#1}}
\title{Economically Protecting Against Covert Timing Channel Attacks\\
(Draft Technical Report)}
\author{Wei Chen, Zhenyun Wen, and David Aspinall\\ 
University of Edinburgh}
\begin{document}
\maketitle
\begin{abstract}
We investigate covert timing channel attacks and mitigations in SDN
(software-defined networking) environment. In particular, we focus on Sneak-Peek
attacks, in which an insider can leak information to an outsider by causing
delays in the packet transmission between outsiders, where the insider and the
outsider are in logically isolated networks but share common physical paths.

We develop a game-theoretic model to formalise the interaction between attackers
and defenders, considering: dynamic time thresholding for adaptive attacks, and
random and selective path-hopping plans for proactive defence. This model is
parameterised by available paths, hopping rate and cost, and traffic volume. Its
equilibrium reveals the optimal strategy (the best hopping plan, the least
hopping rate, and the least effect on normal traffic), for defenders to achieve
a certain level reduction of covert channel capacity, so as to economically
protect against covert timing channel attacks.

We simulate these attacks in a SDN environment implemented in Mininet and test
defence strategies with a variety of parameters. The experiment results validate
the analytic analysis of our game-theoretic model: hopping at a low rate is
enough to significantly mitigate covert timing channel attacks with few
performance loss. 
\end{abstract}
\section{Introduction}
\section{Overview}
\section{A game-theoretic model}
\section{Simulation and evaluation}
\end{document}
